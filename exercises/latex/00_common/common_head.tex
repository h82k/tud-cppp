\providecommand{\additionalOptionsForClass}{}
\documentclass[
  accentcolor=tud1c,	% Color theme for TUD corporate design
  colorbacktitle,		% Titlepage has colored background for title area
  inverttitle,			% Font color of title on titlepage is inverted
  \additionalOptionsForClass
  %german%,
  %twoside
]{tudexercise}
\usepackage{courier}

\parindent0em

% Vertical space of 2ex between paragraphs.
% Allow LaTeX to increase or decrease this by 0.3ex. This enables
% more dynamic typesetting.
\setlength{\parskip}{2ex plus0.3ex minus0.3ex}

\raggedbottom

\usepackage[ngerman]{babel}
\usepackage[utf8]{inputenc}
\usepackage{listings}
\usepackage{booktabs}
\usepackage{amsmath}
\usepackage{algorithm2e}
\usepackage{hyperref}
\usepackage{xspace}
\usepackage{tabularx}
\usepackage{tikz}
\usepackage{cleveref}
\usepackage{numprint}
\usepackage{paralist}
\usepackage{verbatim}
\usepackage{tocloft} % for manipulating the table of contents
\usepackage{graphics}
\usepackage{csquotes} % for \enquote{}
\usepackage{siunitx}

\usetikzlibrary{shapes}
\usetikzlibrary{calc}
\usetikzlibrary{arrows}
\usetikzlibrary{decorations}

\usepackage{pifont}
\newcommand{\cmark}{\ding{51}\xspace}%
\newcommand{\xmark}{\ding{55}\xspace}%

\usepackage{todonotes}
%\usepackage[disable]{todonotes} % Use this line to hide all todos

\definecolor{commentgreen}{RGB}{50,127,50}
\lstloadlanguages{C++,[gnu]make,bash}
\lstset{language=C++}
\lstset{captionpos=b}
\lstset{tabsize=3}
\lstset{breaklines=true}
\lstset{basicstyle=\ttfamily}
\lstset{columns=flexible}
\lstset{keywordstyle=\color{purple}}
\lstset{stringstyle=\color{blue}}
\lstset{commentstyle=\color{commentgreen}}
\lstset{otherkeywords=\#include}
\lstset{showstringspaces=false}
\lstset{keepspaces=true}
\lstset{xleftmargin=1cm}
\lstset{literate=%
	{Ö}{{\"O}}1
	{Ä}{{\"A}}1
	{Ü}{{\"U}}1
	{ß}{{\ss}}2
	{ü}{{\"u}}1
	{ä}{{\"a}}1
	{ö}{{\"o}}1
	{'}{{\textquotesingle}}1
	{~}{$\sim$}1
}

\newcommand{\superscript}[1]{\ensuremath{^{\textrm{#1}}}}
\newcommand{\subscript}[1]{\ensuremath{_{\textrm{#1}}}}

\newcommand{\setHeader}[1]{
\providecommand{\examheadertitle}{TODO: Titel einbinden}
\renewcommand{\examheadertitle}{#1}
\begin{examheader}
    \examheadertitle
\end{examheader}
}

\newcommand{\hints}[1]{
\paragraph*{Hinweise}
	\begin{itemize}
		\setlength{\itemsep}{0pt}
		#1
	\end{itemize}
}

\newcommand{\optional}{\xspace(optional)}
\newcommand{\experimental}{\xspace(experimentell)}

\usepackage{fancybox}
\cornersize{.2} 
\newcommand{\optionaltextbox}{
\bigskip
\begin{center}
\fbox{
\begin{minipage}{.95\textwidth}
Die Klausur kann ohne diese Aufgabe bestanden werden.
Wir empfehlen aber trotzdem sie zu bearbeiten.
\end{minipage}
}
\end{center}
}

\newcommand{\filename}[1]{\texttt{#1}\xspace}
\newcommand{\bspw}{bspw.\xspace}
\newcommand{\dasheisst}{d.\,h.\xspace}
\newcommand{\zB}{z.\,B.\xspace}
\newcommand{\unteranderem}{u.\,a.\xspace}

\newcommand{\winIdeaGroupName}[1]{\textsf{#1}\xspace}
\newcommand{\menuPath}[1]{\emph{#1}\xspace}
\newcommand{\menuSep}[0]{\ensuremath{\to}\;}
\newcommand{\shortcut}[1]{\texttt{#1}\xspace}

\newcommand{\RK}[1]{\todo[]{\textbf{RK:} #1}}
\newcommand{\RKi}[1]{\todo[inline]{\textbf{RK:} #1}}

\newcommand{\ExercisePrefix}[1]{\makebox{\texttt{[#1]}}\xspace}
\newcommand{\ExercisePrefixBasics}{\ExercisePrefix{G}}
\newcommand{\ExercisePrefixMemory}{\ExercisePrefix{S}}
\newcommand{\ExercisePrefixObjectOrientation}{\ExercisePrefix{O}}
\newcommand{\ExercisePrefixAdvanced}{\ExercisePrefix{F}}
\newcommand{\ExercisePrefixEmbeddedC}{\ExercisePrefix{C}}
\newcommand{\ExercisePrefixElevator}{\ExercisePrefix{Z}}

% Displays a hint at the sample solution.
% #1 is the name of the folder in exercises/solutions that contains the solution (e.g., linked_list)
% #2 is similar to #1, but should be valid LaTeX code in text mode (e.g., linked\_list)
\newcommand{\cpppSolutionName}[2]{Ein Lösungsvorschlag für diese Aufgabe liegt im Ordner \href{./exercises/solutions/#1/.}{\texttt{./exercises/solutions/#2}}.}

\newcommand{\cclicense}{
\begin{center}
    \fbox{
        \parbox{0.96\textwidth}
        {
            \begin{minipage}{0.15\textwidth}
                \includegraphics[width=1.0\textwidth]{00_common/cc_license.png}
            \end{minipage}
            \hfill
            \begin{minipage}{0.80\textwidth}
                \begin{small}
                Dieses Werk ist unter einer Creative Commons Lizenz vom Typ Namensnennung - Nicht kommerziell - Keine Bearbeitungen 4.0 International zugänglich. 
                Um eine Kopie dieser Lizenz einzusehen, konsultieren Sie \url{http://creativecommons.org/licenses/by-nc-nd/4.0/} oder 
                wenden Sie sich brieflich an Creative Commons, Postfach 1866, Mountain View, California, 94042 USA.
                \end{small}
            \end{minipage}
        }
    }
\end{center}
}

%% Custom commands for including a listing, based on \lstinputlisting

% The cpppInputNoPageBreakListing command prevents that Latex inserts a page break in between of the listing.  
% If the listing does not fit onto the page anymore, it will be put onto the next page as a whole.
% The first paramer is optional and arguments from lstinputlisting can be provided here.
% The second parameter is the path to the listing.
\newcommand{\cpppInputNoPageBreakListing}[2][] { \begin{minipage}{\textwidth} \lstinputlisting[#1]{#2} \end{minipage} }
% Variant for inline-inserting code
\newcommand{\cpppNoPageBreakListing}[2][] { \begin{minipage}{\textwidth} \begin{lstlisting}[#1]#2 \end{lstlisting}\end{minipage} }

% The cpppInputListing command allows a listing to continue on the next page.
% The first paramer is optional and arguments from lstinputlisting can be provided here.
% The second parameter is the path to the listing.
\newcommand{\cpppInputListing}[2][] { \vspace{1.5ex} \setbox0\vbox{ \lstinputlisting[#1]{#2} } \unvbox0 }
\newcommand{\cpppListing}[2][] { \vspace{1.5ex} \setbox0\vbox{ \begin{lstlisting}[#1]#2 \end{lstlisting} } \unvbox0 }

\newcommand{\chapterstartpage}[1]{%
	\clearpage% Clearpage first, then use pagestyle plain
	\pagestyle{plain}%
	\begingroup
	\vspace*{6cm}
	\centering%
	\begin{tikzpicture}
	%\draw (0,0) \node[rectangle, minimum height=3cm,minimum width=8cm,draw]  { \Huge \textbf{#1}};
	\draw node[rectangle, ultra thick, inner sep= 1cm,draw]{\Huge \textbf{#1}};
	
	\end{tikzpicture}
	 

	%\Huge \textbf{#1}%%
	
	%% Add more space here, do more formatting also here
	\vspace{\baselineskip}%
	%\textbf{#1}%
	\endgroup
	\clearpage
}%



