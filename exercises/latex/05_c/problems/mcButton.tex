\section{\ExercisePrefixEmbeddedC Taster abfragen \optional}

\optionaltextbox

In dieser Aufgabe erweiterst du die vorherige Aufgabe um eine Benutzerinteraktion über den Taster des linken Joysticks (\textbf{Joystick 1}).
Ziel dieser Aufgabe ist es, mithilfe des Tasters die RGB-LED in zwei verschiedenen Szenarien zu kontrollieren.
Im ersten Szenario soll der Taster als Lichtschalter arbeiten: Wird der Taster einmal betätigt, schaltet sich die blaue LED ein; wird der Taster erneut betätigt, schaltet sie sich wieder ab.
Im zweiten Szenario soll die blaue LED solange leuchten, wie der Taster gedrückt gehalten wird.

\begin{enumerate}
\item 
In dieser Aufgabe wirst du mit den Dateien \filename{button.c} und \filename{button.h} arbeiten.

\item 
Zunächst werden wir die nötigen Variablen in \textbf{button.c} deklarieren:
Um den aktuellen Zustand der LED zu speichern wird eine vorzeichenlose 8-Bit-Integer-Variable \textbf{\lstinline|ledStatus|} angelegt.

\item
Implementiere zunächst die Funktion \textbf{\lstinline|initLED()|}.
Diese soll \lstinline|ledStatus| und den Pin der blauen LED initialisieren.
\begin{itemize}
\item 
Der LED-Status soll zu Beginn 0 (= \enquote{aus}) sein.
\item
Der Pin der blauen LED muss als Ausgang konfiguriert werden.
\item 
Das Daten-Register der blauen LED soll entsprechend initialisiert sein, dass die LED ausgeschaltet ist.
\end{itemize}

\item
Implementiere nun die Funktion \textbf{\lstinline|toggleBlueLED()|}.
Diese soll den Status von \lstinline|ledStatus| umkehren:
War der Wert zuvor \lstinline|1|, soll er danach \lstinline|0| sein und umgekehrt.
Der aktuelle Status der LED soll mit \textbf{\lstinline|setBlueLED(uint8_t status)|} gesetzt werden.

\item 
Implementiere nun die Funktion \textbf{\lstinline|isButtonPressed|}, die zurück gibt, ob der Taster gerade gedrückt ist.
Beachte, dass der Taster durch den Pull-Up-Widerstand genau dann gedrückt ist, wenn am Pin ein niedriger Pegel (\lstinline|0|) anliegt.
Ein Beispiel für die Abfrage des Tasters findest du in \Cref{lst:IO_example}.

\item
Implementiere nun mithilfe der zuvor erstellten Hilfsfunktionen die Hauptfunktionen \textbf{ButtonToggleBlueLED()} und \textbf{ButtonHoldBlueLEDOn()}.
Die erste soll dem Button die Funktion eines Lichtschalters geben und die zweite soll die LED zum Leuchten bringen, solange der Taster gedrückt gehalten wird. 

\end{enumerate}
