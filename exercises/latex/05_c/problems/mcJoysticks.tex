\section{\ExercisePrefixEmbeddedC Joysticks abfragen \optional}
In diesem Abschnitt nutzen wir die Funktionen der vorigen Aufgabe, um die analogen Werte der Joysticks auszugeben. Darauf aufbauend entwickelst du eine Aufgabe, die mithilfe des Joysticks die Farbe der RGB-LED verändert.
Die zu implementierenden Funktionen befinden sich in der Datei \filename{joystick.c}.
Solltest du die vorherige Aufgabe nicht (vollständig) bearbeitet haben, kannst du auf die Musterlösung in der Datei \filename{display\_s.c} zurückgreifen.
Um diese statt deiner eigenen Lösung zu nutzen, hängst du an jeden Funktionsnamen das Suffix \filename{\_s} an (\bspw \lstinline|writeChar_s| statt \lstinline|writeChar|).

\subsection{Analoge Werte auf dem Display anzeigen}

Jeder Joystick besitzt zwei analoge Leitungen, welche die X- oder Y-Position des Steuerknüppels auslesen.
Die analogen Werte entsprechend dabei der Spannung des jeweils Drehpotentiometers, welche zwischen \SI{0}{\volt} und \SI{5}{\volt} liegt.
Die Spannungswerte der Joysticks werden durch den Analog-Digital-Wandler des Microcontrollers auf einen 8-Bit-Wert abgebildet (Wertebereich: 0 bis 255).
Die Aufteilung der Wertebereiche sowie die Orientierung der X- und Y-Richtung sind in \Cref{fig:jostickValues} dargestellt.
%
\begin{figure}
    \begin{centering}
        \includegraphics[width=.4\textwidth]{./05_c/figures/joystickValues.png}
        \caption{Wertebereich der Joysticks}
        \label{fig:jostickValues}
    \end{centering}
\end{figure}
%
Die Aufteilung ist nicht gleichmäßig.
Stattdessen ergibt sich in Neutralstellung der Wert $(190,190)$.
Joystick 1 ist mit den analogen Anschlüssen AN16 (X) und AN19 (Y) und Joystick 2 ist mit AN13 (X) und AN23 (Y) verbunden. 

\begin{enumerate}
\item
Implementiere die Funktion \lstinline|printValues|, welche über Zeiger auf die anlogen Leitungen AN13, AN16, AN19 und AN23 die Werte der Joysticks ausliest und diese auf dem Bildschirm ausgibt.  
Nutze dazu die Funktionen in \Cref{tab:joystickInfo} und folgendes Codefragment, mithilfe dessen man die Analog-Kanäle ausliest:
\begin{lstlisting}
// #include "analog.h"
uint8_t analog11;
uint8_t analog12;
uint8_t analog13;
uint8_t analog16;
uint8_t analog19;
uint8_t analog23;
uint8_t analog17;
getAnalogValues(&analog11, &analog12, &analog13, &analog16, &analog17, &analog19, &analog23);
\end{lstlisting}
%
\begin{table}[]
    \centering
    \caption{Wichtige Funktionen und Variablen für die Verwendung der Joysticks}
    \label{tab:joystickInfo}
    \begin{tabular}{p{7cm}p{7cm}}
        \toprule
        \textbf{Funktionen/Variablen} & \textbf{Beschreibung} \\
        \midrule
        \lstinline|void setCursor(480,320)| & Setzt den Cursor auf den Anfang \\
        \lstinline|void writeTextln(char *text)| & Schreibt \lstinline|text| auf das Display und verschiebt den Cursor mit Zeilensprung \\
        \lstinline|void writeText(char *text)| & Schreibt \lstinline|text| auf das Display und verschiebt den Cursor ohne Zeilensprung \\
        \lstinline|void write3Digits8Bit(uint8_t *number)| & Schreibt die per Pointer referenzierte Zahl \lstinline|number| an die Position des Cursors und verschiebt den Cursor\\
        \bottomrule
    \end{tabular}
\end{table}

\item 
Um fortlaufend, die Positionsdaten des Joysticks auszugeben, rufe \lstinline|printValues| in einer Schleife auf:
\begin{minipage}{\textwidth}
\begin{lstlisting}[]
#include "init.h"

int main(){
  initBoard();
  while(1) {
    printValues();
    delay(1000);
  }
}
\end{lstlisting}\end{minipage} 
\end{enumerate}





\subsection{LED mit Joystick 1 kontrollieren}
In dieser Aufgabe soll die Funktion \lstinline|controlLEDs| geschrieben werden, um die Farbe der RGB-LED durch die Bewegung des Joystick 1 nach links oder rechts zu verändern.
\Cref{tab:controlLED} zeigt die verschiedenen möglichen Wertebereiche mit den anzusteuernden LEDs.
%
\begin{table}[]
    \centering
    \caption{Anzeigebereiche der LEDs}
    \label{tab:controlLED}
    \begin{tabular}{llllr}
        \toprule
        \textbf{Position des Joystick} & \textbf{LED-Farbe} & \textbf{Werebereich AN19} & \textbf{Ausgabe-Pin} & \textbf{Ausgabe-Analog-Kanal}\\
        \midrule
        Links & Grün & 255 \dots 200 & B2 & 8\\
        Mitte & Blau & 200 \dots 180 & 18 & 18\\
        Rechts & Rot & 180 \dots 0 & 1A & 10\\
        \bottomrule
    \end{tabular}
\end{table}
%
\begin{enumerate}
\item
Implementiere zunächst die Hilfsfunktion \lstinline|controlLEDsInit|, welche die Leitungen der RGB-LEDs initialsiert. Die analogen Kanäle der LEDs sollen ausgeschaltet werden.
Definiere die Pins der LEDs als Ausgänge und initialisiere sie mit \lstinline|1u| (= \enquote{aus}). 

\item
Implementiere nun die Funktion \lstinline|controlLEDs|, welche die Position der X-Achse des Joystick über den analogen Kanal AN19 ausliest und die Farbe der RGB-LED gemäß Tabelle \ref{tab:controlLED} verändert. 

\item 
Dein Testcode sollte in etwas wie folgt aussehen:

\begin{minipage}{\textwidth}
\begin{lstlisting}[]
#include "init.h"

int main(){
  initBoard();
  controlLedsInit();
  while(1) {
    controlLeds();
    delay(1000);
  }
}
\end{lstlisting}\end{minipage} 
\end{enumerate}
