\section{A/D-Wandler}
Schreibe ein Programm, das den Spannungswert von \emph{AN1} (linker Schieberegler) auf der linken Siebensegmentanzeige und den Spannungswert von \emph{AN2} (rechter Schieberegler) auf der rechten Siebensegmentanzeige ausgibt.
Skaliere dazu den resultierenden Wertebereich (0 bis 255) auf 0 bis 9.

Mache dich mit folgenden Funktionen vertraut und verwende sie zur Initialisierung und anschließender Verwendung des A/D-Wandlers:
\lstinputlisting{problems/listings/adconv.c}

\hints{
	\item Einige Funktionen aus \ref{exercise7SegmentUtil} kannst du hier wiederverwenden.
	\item Der Wert für \lstinline{ADSR} besteht aus 16 Bits (0110 11xx xxxy yyyy\textsubscript{b}), wobei xxxxx für den Startkanal der Konvertierung und yyyyy für den Endkanal steht.
	Für unsere Zwecke nehmen diese beiden 5\,Bit Blöcke immer entweder 00001\textsubscript{b} = 1 (AN1) oder 00010\textsubscript{b} = 2 (AN2) an.
	Daher lässt sich der \lstinline{ADSR} auch wie folgt über einen binären Shift und zwei Addition bestimmen:
	$$\text{\lstinline{ADSR}} = 0110110000000000\textsubscript{b} + (xxxxx\textsubscript{b} << 5) + yyyyy\textsubscript{b} = 0x6c00 + (x << 5) + y$$
}
