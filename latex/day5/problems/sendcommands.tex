\subsection*{Befehle senden}
Ein Befehl an das Display besteht grundsätzlich aus zwei Phasen:
\begin{enumerate}
\item Setzen der entsprechenden Pins (meist \lstinline{LCD_PIN_DI}, \lstinline{LCD_PORT_DB}).
	Eventuell selektieren der Displayhälfte über \lstinline{LCD_PIN_CS1, LCD_PIN_CS2}.

\item
Senden des \lstinline{enable}-Signals.
Setze dazu den Enable-Pin auf 1, warte kurz, setze den Pin wieder auf 0 und warte erneut kurz.
Verwende als Warteintervall die Konstante \lstinline{LCD_T}.

Es empfiehlt sich, für dieses Signal eine eigene Funktion zu schreiben (\lstinline{void lcd_sendEnable(void)}).
\end{enumerate}
