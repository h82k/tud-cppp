\section{Funktionales Programmieren}

In dieser Aufgabe werden wir uns mit Funktionen aus der funktionalen Programmierung beschäftigen.
Diese sind \lstinline{map}, \lstinline{filter} und \lstinline{reduce}. \\

Der Ablauf ist wie folgt:
\begin{itemize}
    \item Wir werden im folgenden erst einmal einzeln auf die Funktionen eingehen.
    \item Danach wird es eure Aufgabe sein, diese Funktionen nachzuprogrammieren.
    \item Da wir diese Funktionen danach verwenden wollen, werden wir dafür passende Funktionen programmieren.
    \item Diese werden wir darauf folgend verbessert.
\end{itemize}

\subsection{Erklärung \lstinline{map}, \lstinline{filter} und \lstinline{reduce}}

Arbeitet man auf iterierbaren Sequenzen, ist dies fast immer mit Schleifen über die Sequenz verbunden.
Die drei genannten Funktionen vereinfachen uns hierbei dir Arbeit.
Hierzu ein Beispiel.

Haben wir ein Vektor von Integern und wir wollen jedes Element quadrieren, endet dies meist in dem folgenden Programmcode:

\begin{lstlisting}
std::vector<int> numbers = { 1, 2, 3, 4, 5 };

for (std::vector<int>::iterator it = numbers.begin(); it != numbers.end(); ++it) {
    *it = square(*it); // squaring element pointed to by it
}
\end{lstlisting}

Die Idee von der Funktion \lstinline{map} ist es genau dies zu vereinfachen.
Sie bekommt den Startiterator, der Sequenz und den Enditerator.
Außerdem bekommt sie einen Funktionspointer als Parameter und ruft diese Funktion auf jedes Element der iterierbaren Sequenz auf.

\begin{lstlisting}
std::vector<int> numbers = { 1, 2, 3, 4, 5 };

map(numbers.begin(), numbers.end(), square);
\end{lstlisting}

\lstinline{filter} funktioniert analog, indem sie einen Funktionspointer auf eine Funktion erhält, die einen \lstinline{bool} zurück gibt.
Auf alle Elemente wird diese Funktion aufgerufen und entfernt die Elemente, für die die Funktion \lstinline{true} zurückgibt. \\

Und auch \lstinline{reduce} hat eine ähnliche Verwendung.
Es schrumpft eine Sequenze zu einem Element zusammen.
Zum Beispiel bildet es die Summe eines Vektors.

\subsection{Programmieren der Funktionen}

Nun werden wird es deine Aufgabe sein, die drei Funktionen nachzuprogrammieren.
Hierbei geht es erstmal darum ein funktionierendes Gerüst der Methoden zu erstellen, anstatt perfekt generische Algorithmen zu erhalten.
Darum wird sich im Laufe der Aufgabe gekümmert.

