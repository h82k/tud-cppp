\section{Generische Vektor-Implementation}
Erinnere dich an die Klasse \texttt{Vector3} aus dem ersten Praktikumstag. Diese hat den Datentyp \texttt{double} für die einzelnen Komponenten verwendet. Schreibe die Klasse so um, dass der Datentyp der Komponenten durch einen Template-Parameter angegeben werden kann.
Füge dafür der Klasse \texttt{Vector3} einen Template-Parameter hinzu und ersetze jedes Aufkommen von \texttt{double} mit dem Template-Parameter.
Vergiss nicht, die Implementation in den Header zu verschieben, da der Compiler die Definition einer Klasse kennen muss, um beim Einsetzen des Template-Parameters den richtigen Code zu generieren.

Verbessere außerdem die Effizienz und Sauberkeit der \texttt{Vector3}-Klasse, in dem du die Parameterübergabe in den entsprechenden Methoden auf \texttt{const} Referenzen umstellst und alle Getter als \texttt{const} deklarierst.
