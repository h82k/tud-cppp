\section{Standard Container} 
In dieser Aufgabe werden wir den Umgang mit den Containern \texttt{std::vector} und \texttt{std::list} aus der Standard Template Library üben.
Es ist sinnvoll, wenn du während der Übung eine C++-Referenz zum Nachschlagen bereithältst, z.B. \url{http://www.cplusplus.com/}.
Schaue dir auch die Vorlesungsfolien genau an, da diese nützliche Codebeispiele enthalten.

Die Klasse \texttt{std::list} stellt eine verkettete Liste dar, bei der man an beliebiger Stelle Elemente effizient löschen und hinzufügen kann. \texttt{std::vector} stellt ähnliche Funktionen bereit, allerdings liegen hier die Elemente in einem einzigen, zusammenhängenden Speicherbereich, der neu alloziert und kopiert werden muss, wenn seine aktuelle Kapazität überschritten wird.
Auch müssen viele Elemente verschoben werden, wenn der Vektor in der Mitte oder am Anfang modifiziert wird.
Der große Vorteil von \texttt{std::vector} ist der \emph{wahlfreie Zugriff}, d.h. man kann auf beliebige Elemente mit konstantem Aufwand zugreifen.


\begin{enumerate}
\item 
Schreibe zunächst eine Funktion \texttt{template<typename T> void print(const T \&t)}, die beliebige Standardcontainer auf die Konsole ausgeben kann, die Integer speichern und Iteratoren unterstützen.
Nutze dazu die Funktion \texttt{copy()} sowie die Klasse \texttt{std::ostream\_iterator<int>}, um den entsprechenden \texttt{OutputIterator} zu erzeugen.

\item
Lege ein \texttt{int}-Array an und initialisiere es mit den Zahlen 1 bis 5.
Lege nun einen \texttt{std::vector<int>} an und initialisiere ihn mit den Zahlen aus dem Array.

\item
Lege eine Liste \texttt{std::list<int>} an und initialisiere diese mit dem zweiten bis vierten Element des Vektors.
\textbf{Tipp}: Du kannst auf Iteratoren eines Vektors (genauso wie auf Zeiger) Zahlen addieren, um diese zu verschieben.

\item
Füge mittels \texttt{std::list<T>::insert()} das letzte Element des Vektors an den Anfang der Liste hinzu.

\item
Lösche alle Elemente des Vektors mit einem einzigen Methodenaufruf.

\item
Mittels \texttt{remove\_copy\_if()} kann man Elemente aus einem Container in einen anderen kopieren und dabei bestimmte Elemente löschen lassen.
Nutze diese Funktion, um alle Elemente, die kleiner sind als 4, aus der Liste in den Vektor zu kopieren. Beachte, dass \texttt{remove\_copy\_if()} keine neuen Elemente an den Container anhängt, sondern lediglich Elemente von der einen Stelle zur anderen elementweise durch Erhöhen des \texttt{OutputIterator} kopiert.

Deshalb kannst du \texttt{vec.end()} \textbf{nicht} als \texttt{OutputIterator} nehmen, da dieser "{}hinter"{} das letzte Element zeigt und weder dereferenziert noch inkrementiert werden darf. Nutze stattdessen die Methode \texttt{back\_inserter()}, um einen Iterator zu erzeugen, der neue Elemente an den Vektor anhängen kann.
\end{enumerate}
