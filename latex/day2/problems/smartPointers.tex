\section{Smart Pointers}
In dieser Aufgabe werden wir uns mit der Benutzung von Smart Pointers vertraut machen. Dazu werden wir die Smart Pointer Klassen \lstinline{std::shared\_ptr} und \lstinline{std::weak\_ptr} verwenden. Binde hierfür den Systemheader \lstinline{memory} ein.


\subsection{}
Erstellen eine Klasse \lstinline{TreeNode}, die einen Knoten eines Binärbaums darstellt.
Jeder Knoten hat einen Inhalt vom Typ \lstinline{int} sowie einen Zeiger auf seine beiden Kindknoten.
Statt \glqq roher\grqq{} Zeiger verwenden wir Smart Pointers, die das Speichermanagement übernehmen.
Dadurch wird es nicht nötig sein, Kindknoten manuell zu löschen.
Sie werden automatisch entfernt, sobald der Wurzelknoten gelöscht ist und keine Zeiger mehr auf den Kindknoten zeigen.

\begin{lstlisting}
#include <memory>
class TreeNode;

typedef std::shared_ptr<TreeNode> TreeNodePtr;	// typedef for better readability

class TreeNode {
public:
	/** create a new tree node and make it shared */
	static TreeNodePtr createNode(int content, TreeNodePtr left = TreeNodePtr(), TreeNodePtr right = TreeNodePtr());
	~TreeNode();
private:
	TreeNode(int content, TreeNodePtr left, TreeNodePtr right);	// create a tree node
	TreeNodePtr leftChild, rightChild;									// left and right child
	int content;																// node content
};
\end{lstlisting}

Der Konstruktor von \lstinline{TreeNode} privat, weil nur die Smart Pointer die Verantwortung für die Lebenszeit eines Objektes übernehmen sollen und bestimmen, wann es gelöscht wird.
Würde man \lstinline{TreeNode}-Objekte direkt auf dem Stack anlegen, kann es passieren, dass der Objektdestruktor mehrmals aufgerufen wird -- einmal vom Smart Pointer und einmal beim Verlassen der Funktion.
Ebenso sollten wir keine Rohzeiger auf das Objekt erzeugen, da diese das Speichermanagement der Smart Pointer umgehen.
Stattdessen stellen wir eine statische Methode bereit, um \lstinline{TreeNode}-Objekte auf dem Heap zu erzeugen und diese direkt einem Smart Pointer zu übergeben.

Implementiere den Konstruktor, Destruktor sowie \lstinline{createNode}.
Der Konstruktor sollte die Attribute entsprechend initialisieren.
Schreibe auch eine Textausgabe, die den Zeitpunkt der Erzeugung eines \lstinline{TreeNode}s deutlich macht.
Der Destruktor braucht die Kindknoten nicht zu löschen, da dies bei der Zerstörung des Elternknotens automatisch geschieht.
Füge auch hier eine Textausgabe ein, die die Zerstörung des Objekts sichtbar macht.

Das Schlüsselwort \textbf{static} sowie die Default-Parameter müssen bei der Implementation der Methode ausgelassen werden.
Der Smart Pointer für die Rückgabe wird mit einem Zeiger auf ein \lstinline{TreeNode}-Objekt initialisiert. Somit lautet der Methodenrumpf

\begin{lstlisting}
TreeNodePtr TreeNode::createNode(int content, TreeNodePtr left, TreeNodePtr right) {
	return TreeNodePtr(new TreeNode(...));
}
\end{lstlisting}

\hints{
	\item Über \lstinline{std::make\_shared(<Konstruktorparameter>)} kann man auch einen Smart Pointer erhalten.
}

\subsection{}
Teste, ob die einzelnen Knoten tatsächlich gelöscht werden, sobald kein Zeiger mehr auf den Elternknoten zeigt.
Erstelle dafür einen kleinen Baum:

\begin{lstlisting}
TreeNodePtr node = TreeNode::createNode(1, TreeNode::createNode(2), TreeNode::createNode(3));
\end{lstlisting}

Führe das Programm aus und beobachte die Ausgabe.
Sobald \lstinline{main} verlassen wird, wird der Zeiger \lstinline{node} gelöscht, und somit auch das dahinterliegende \lstinline{TreeNode}-Objekt mit all seinen Kindknoten.

Um ganz sicher zu gehen, dass der Baum tatsächlich beim Löschen des letzten Zeigers zerstört wurde und nicht etwa durch das Beenden des Programms, kannst du \lstinline{node} mit einem anderen Baum überschreiben.
Füge in diesem Fall am Ende des Programms eine Textausgabe hinzu, damit ersichtlich wird, dass der erste Baum noch vor Verlassen der \lstinline{main} gelöscht wurde.

\subsection{}
Nun wollen wir \lstinline{TreeNode} so erweitern, dass jeder Knoten Kenntnisse über seinen Elternknoten besitzt.
Füge das Attribut

\begin{lstlisting}
	TreeNodePtr parent;		// parent node
\end{lstlisting}

hinzu.
Da der Elternknoten beim Erzeugen eines \lstinline{TreeNode}s undefiniert ist, brauchst du den Konstruktor nicht zu ändern. \lstinline{parent} wird dann automatisch mit \lstinline{NULL} initialisiert.

Implementiere die folgende Methode, die einem Knoten seinen Elternknoten zuweist:

\begin{lstlisting}
	void setParent(const TreeNodePtr &p);		// set parent of this node
\end{lstlisting}

\hints{
	\item \lstinline{p} wird in diesem Fall nur deshalb als \lstinline{const} Referenz übergeben, da es verhältnismäßig aufwändig ist, einen Smart Pointer zu kopieren.
	Beachte, dass im obigen Fall der Smart Pointer selbst \lstinline{const} ist, und nicht das Objekt, worauf er zeigt.
}

Jetzt muss noch \lstinline{createNode()} modifiziert werden, sodass \lstinline{setParent()} auf den Kindknoten aufgerufen wird.
Da ein Smart Pointer den \lstinline{operator*} und den \lstinline{operator->} überladen hat, lässt er sich syntaktisch wie ein normaler Zeiger benutzen. Um zu überprüfen, ob ein Smart Pointer auf ein Objekt zeigt, kann dieser implizit nach \lstinline{bool} gecastet werden. Somit lautet die neue Implementation von \lstinline{createNode()}:

\begin{lstlisting}
TreeNodePtr TreeNode::createNode(int content, TreeNodePtr left, TreeNodePtr right) {
	TreeNodePtr node(new TreeNode(content, left, right));
	if (left) {
		left-> ... ; // set parent node
	}
	if (right) {
		right-> ... ; // set parent node
	}
	return node;
}
\end{lstlisting}

\subsection{}
Teste deine Implementation.
Du brauchst dazu in \lstinline{main} nichts zu ändern.

Erschreckenderweise siehst du nun, dass überhaupt keine \lstinline{TreeNode}-Objekte mehr gelöscht werden.
Die Ursache dafür ist die zirkuläre Abhängigkeit zwischen Kind- und Elternknoten.
Denn selbst wenn sie keine Zeiger auf den Wurzelknoten eines Baumes haben, verweisen die Kindknoten noch immer darauf.

Um dieses Problem zu lösen, müssen die Verweise zum Elternknoten \emph{schwach} (weak) sein.
Ein Knoten darf gelöscht werden, wenn nur noch schwache Zeiger (oder keine) auf ihn verweisen.
Verwende dazu \lstinline{std::weak\_ptr} und erstelle ein neues \lstinline{typedef} für einen schwachen \lstinline{TreeNode} Smart Pointer:

\begin{lstlisting}
typedef std::weak_ptr<TreeNode> TreeNodeWeakPtr;
\end{lstlisting}

Ändere nun den Typ von \lstinline{parent} auf \lstinline{TreeNodeWeakPtr}.
Es müssen keine weiteren Änderungen gemacht werden, da starke Zeiger (\lstinline{shared\_ptr}) implizit in schwache Zeiger (\lstinline{weak\_ptr}) umgewandelt werden können.

\hints{
	\item \textbf{Faustregel}: Wenn ein Objekt ein anderes kontrolliert oder enthält, verwende Shared Pointer vom Container/Besitzer zum anderen Objekt und einen Weak Pointer für die umgekehrte Richtung.
}

\subsection{}
Teste deine Implementation.
Nun sollte sich \lstinline{TreeNode} wie gewünscht verhalten.
